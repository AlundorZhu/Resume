\documentclass[10pt, letterpaper]{article}

% Packages:
\usepackage[
    ignoreheadfoot, % set margins without considering header and footer
    top=2 cm, % seperation between body and page edge from the top
    bottom=2 cm, % seperation between body and page edge from the bottom
    left=2 cm, % seperation between body and page edge from the left
    right=2 cm, % seperation between body and page edge from the right
    footskip=1.0 cm, % seperation between body and footer
    % showframe % for debugging 
]{geometry} % for adjusting page geometry
\usepackage{titlesec} % for customizing section titles
\usepackage{tabularx} % for making tables with fixed width columns
\usepackage{array} % tabularx requires this
\usepackage[dvipsnames]{xcolor} % for coloring text
\definecolor{primaryColor}{RGB}{0, 0, 0} % define primary color
\usepackage{enumitem} % for customizing lists
\usepackage{fontawesome5} % for using icons
\usepackage{amsmath} % for math
\usepackage[
    pdftitle={John Doe's CV},
    pdfauthor={John Doe},
    pdfcreator={LaTeX with RenderCV},
    colorlinks=true,
    urlcolor=primaryColor
]{hyperref} % for links, metadata and bookmarks
\usepackage[pscoord]{eso-pic} % for floating text on the page
\usepackage{calc} % for calculating lengths
\usepackage{bookmark} % for bookmarks
\usepackage{lastpage} % for getting the total number of pages
\usepackage{changepage} % for one column entries (adjustwidth environment)
\usepackage{paracol} % for two and three column entries
\usepackage{ifthen} % for conditional statements
\usepackage{needspace} % for avoiding page brake right after the section title
\usepackage{iftex} % check if engine is pdflatex, xetex or luatex

% Ensure that generate pdf is machine readable/ATS parsable:
\ifPDFTeX
    \input{glyphtounicode}
    \pdfgentounicode=1
    \usepackage[T1]{fontenc}
    \usepackage[utf8]{inputenc}
    \usepackage{lmodern}
\fi

\usepackage{charter}

% Some settings:
\raggedright
\AtBeginEnvironment{adjustwidth}{\partopsep0pt} % remove space before adjustwidth environment
\pagestyle{empty} % no header or footer
\setcounter{secnumdepth}{0} % no section numbering
\setlength{\parindent}{0pt} % no indentation
\setlength{\topskip}{0pt} % no top skip
\setlength{\columnsep}{0.15cm} % set column seperation
\pagenumbering{gobble} % no page numbering

\titleformat{\section}{\needspace{4\baselineskip}\bfseries\large}{}{0pt}{}[\vspace{1pt}\titlerule]

\titlespacing{\section}{
    % left space:
    -1pt
}{
    % top space:
    0.3 cm
}{
    % bottom space:
    0.2 cm
} % section title spacing

\renewcommand\labelitemi{$\vcenter{\hbox{\small$\bullet$}}$} % custom bullet points
\newenvironment{highlights}{
    \begin{itemize}[
        topsep=0.10 cm,
        parsep=0.10 cm,
        partopsep=0pt,
        itemsep=0pt,
        leftmargin=0 cm + 10pt
    ]
}{
    \end{itemize}
} % new environment for highlights


\newenvironment{highlightsforbulletentries}{
    \begin{itemize}[
        topsep=0.10 cm,
        parsep=0.10 cm,
        partopsep=0pt,
        itemsep=0pt,
        leftmargin=10pt
    ]
}{
    \end{itemize}
} % new environment for highlights for bullet entries

\newenvironment{onecolentry}{
    \begin{adjustwidth}{
        0 cm + 0.00001 cm
    }{
        0 cm + 0.00001 cm
    }
}{
    \end{adjustwidth}
} % new environment for one column entries

\newenvironment{twocolentry}[2][]{
    \onecolentry
    \def\secondColumn{#2}
    \setcolumnwidth{\fill, 6 cm}
    \begin{paracol}{2}
}{
    \switchcolumn \raggedleft \secondColumn
    \end{paracol}
    \endonecolentry
} % new environment for two column entries

\newenvironment{threecolentry}[3][]{
    \onecolentry
    \def\thirdColumn{#3}
    \setcolumnwidth{, \fill, 4.5 cm}
    \begin{paracol}{3}
    {\raggedright #2} \switchcolumn
}{
    \switchcolumn \raggedleft \thirdColumn
    \end{paracol}
    \endonecolentry
} % new environment for three column entries

\newenvironment{header}{
    \setlength{\topsep}{0pt}\par\kern\topsep\centering\linespread{1.5}
}{
    \par\kern\topsep
} % new environment for the header

\newcommand{\placelastupdatedtext}{% \placetextbox{<horizontal pos>}{<vertical pos>}{<stuff>}
  \AddToShipoutPictureFG*{% Add <stuff> to current page foreground
    \put(
        \LenToUnit{\paperwidth-2 cm-0 cm+0.05cm},
        \LenToUnit{\paperheight-1.0 cm}
    ){\vtop{{\null}\makebox[0pt][c]{
        \small\color{gray}\textit{Last updated in September 2024}\hspace{\widthof{Last updated in September 2024}}
    }}}%
  }%
}%

% save the original href command in a new command:
\let\hrefWithoutArrow\href

% new command for external links:


\begin{document}
    \newcommand{\AND}{\unskip
        \cleaders\copy\ANDbox\hskip\wd\ANDbox
        \ignorespaces
    }
    \newsavebox\ANDbox
    \sbox\ANDbox{$|$}

    \begin{header}
        \fontsize{25 pt}{25 pt}\selectfont Ziqiang "Joe" Zhu

        \vspace{5 pt}

        \normalsize
        \mbox{Davis, CA}%
        \kern 5.0 pt%
        \AND%
        \kern 5.0 pt%
        \mbox{\hrefWithoutArrow{mailto:ziqzhu@ucdavis.edu}{ziqzhu@ucdavis.edu}}%
        \kern 5.0 pt%
        % \AND%
        % \kern 5.0 pt%
        % \mbox{\hrefWithoutArrow{tel:+1-669-388-23-99}{669 388 23 99}}%
        % \kern 5.0 pt%
        % \AND%
        % \kern 5.0 pt%
        % \mbox{\hrefWithoutArrow{https://yourwebsite.com/}{yourwebsite.com}}%
        % \kern 5.0 pt%
        \AND%
        \kern 5.0 pt%
        \mbox{\hrefWithoutArrow{https://linkedin.com/in/joe-zhu-032153259}{linkedin.com/in/joe-zhu-032153259}}%
        \kern 5.0 pt%
        \AND%
        \kern 5.0 pt%
        \mbox{\hrefWithoutArrow{https://github.com/AlundorZhu}{github.com/AlundorZhu}}%
    \end{header}

    \vspace{5 pt - 0.3 cm}


    % \section{Welcome to RenderCV!}



        
    %     \begin{onecolentry}
    %         \href{https://rendercv.com}{RenderCV} is a LaTeX-based CV/resume version-control and maintenance app. It allows you to create a high-quality CV or resume as a PDF file from a YAML file, with \textbf{Markdown syntax support} and \textbf{complete control over the LaTeX code}.
    %     \end{onecolentry}

    %     \vspace{0.2 cm}

    %     \begin{onecolentry}
    %         The boilerplate content was inspired by \href{https://github.com/dnl-blkv/mcdowell-cv}{Gayle McDowell}.
    %     \end{onecolentry}


    
    % \section{Quick Guide}

    % \begin{onecolentry}
    %     \begin{highlightsforbulletentries}


    %     \item Each section title is arbitrary and each section contains a list of entries.

    %     \item There are 7 unique entry types: \textit{BulletEntry}, \textit{TextEntry}, \textit{EducationEntry}, \textit{ExperienceEntry}, \textit{NormalEntry}, \textit{PublicationEntry}, and \textit{OneLineEntry}.

    %     \item Select a section title, pick an entry type, and start writing your section!

    %     \item \href{https://docs.rendercv.com/user_guide/}{Here}, you can find a comprehensive user guide for RenderCV.


    %     \end{highlightsforbulletentries}
    % \end{onecolentry}

    \section{Education}



        
        \begin{twocolentry}{
            Sept 2021 – Present
        }
            \textbf{University of California, Davis}
            
        \end{twocolentry}

        \vspace{0.10 cm}
        \begin{onecolentry}
            \begin{highlights}
                \item \textbf{M.S. Computer Science}, Machine Learning \& Computer Vision
                \item \textbf{B.S. Computer Science}, Minor in English \textbf{GPA: } 3.7/4.0% (\href{https://example.com}{a link to somewhere})
            \end{highlights}
            \textbf{Related Courses: } Machine Learning \& Discovery, Artificial Intelligence, Linear Algebra, Statistics, Operating System, Computer Architecture, Computer Network, Gameplay Programming, Computer Security
        \end{onecolentry}



    
    \section{Experience}



        
        \begin{twocolentry}{
            June 2024 – Present
        }
            \textbf{Research Assistant}, CS Department -- Davis, CA\end{twocolentry}

        \vspace{0.10 cm}
        \begin{onecolentry}
            \begin{highlights}
                \item Collected and curated datasets; explored generating synthetic images, using both off-the-shelf and custom labeler to improve efficiency.
                \item Trained computer vision models for keypoint using pytorch/ultralytics and object detection with keras/tensorflow, achieving robust performance across diverse clinical datasets.
                \item Designed and validated an algorithm for real-time clinical test analysis, reaching $>$99\% agreement with physician assessments on rehabilitation tasks.
                \item Extracted fine-grained motion features beyond human perceptual limits, enabling deeper insights into stroke rehabilitation and recovery patterns. 
               \item Currently implementing the algorithm on mobile devices for real-time use.  \href{https://github.com/AlundorZhu/CMORE-app}{\textbf{IOS demo app}} in progress
            \end{highlights}
        \end{onecolentry}

        \vspace{0.2 cm}

        \begin{twocolentry}{
            Sep 2025 – Present
        }
            \textbf{Teaching assistant}, Computer Architecture -- Davis, CA\end{twocolentry}

        \vspace{0.10 cm}
        \begin{onecolentry}
            \begin{highlights}
                \item Created and graded assignments, held office hours, and supported students with course material.
                \item Developed auto grader with clear test message
                \item Help improve the course by leveraging Professor's learning object and students' feedback

            \end{highlights}
        \end{onecolentry}


        \vspace{0.2 cm}

        \begin{twocolentry}{
            Jan 2023 – Present
        }
            \textbf{Club Mentor/Leadership}, Cyclone Robosub -- Davis, CA\end{twocolentry}

        \vspace{0.10 cm}
        \begin{onecolentry}
            \begin{highlights}
                \item Semi-final finish at RoboSub 2025 competition, outperformed UC Berkeley
                \item Mentoring the vision system of the autonomous underwater vehicle
                \item Developed custom video recording, labeling, training tools for ML 
                \item Strong background in collaborating with people of different discipline

            \end{highlights}
        \end{onecolentry}


        \vspace{0.2 cm}

        \begin{twocolentry}{
            June 2024 – Present
        }
            \textbf{Unitrans Driver}, ASUCD -- Davis, CA\end{twocolentry}

        \vspace{0.10 cm}
        \begin{onecolentry}
            \begin{highlights}
                \item Responsible for the safe and efficient operation of a heavy duty public transit bus carrying on average 50 customers per hour per vehicle
                \item Acute awareness and training in time management
                \item Member of a large team working collaborative

            \end{highlights}
        \end{onecolentry}



    
    \section{Publications}



        
        \begin{samepage}
            \begin{onecolentry}
                \textbf{Markerless Motion Capture Enhances Clinical Assessments: Preliminary Validation with the Box and Blocks Test} 2025 International Conference On Rehabilitation Robotics (ICORR), Chicago, IL, USA, 2025, pp. 1506-1511, doi: \href{https://doi.org/10.1109/ICORR66766.2025.11063098}{10.1109/ICORR66766.2025.11063098.}
            \end{onecolentry}

            \vspace{0.10 cm}
            
            \begin{onecolentry}
                \mbox{Andria Farrens}, \mbox{Vicky Chan}, \mbox{Luis Garcia-Fernandez}, \mbox{\textbf{\textit{Ziqiang “Joe” Zhu}}},

                \vspace{0.10 cm}
                
        \end{onecolentry}
        \end{samepage}


    
    \section{Projects}



        
        \begin{twocolentry}{
            \href{https://github.com/Cyclone-Robosub/Labeler}{github.com/Cyclone-Robosub/Labeler}
        }
            \textbf{Video labeling Tool}\end{twocolentry}

        \vspace{0.10 cm}
        \begin{onecolentry}
            \begin{highlights}
                \item To improve the efficiency of labeling custom dataset. I Developed a video labeler that with a single click tracks the object through the video and export bounding box for each frame in standard format.  
                \item Tools Used: Python, SAM2, Tkinter, Image processing, Segmentation, COCO Format. 
            \end{highlights}
        \end{onecolentry}


        \vspace{0.2 cm}

        \begin{twocolentry}{
            \href{https://github.com/AlundorZhu/CMORE-app}{github.com/AlundorZhu/CMORE-app}
        }
            \textbf{CMORE demo app}\end{twocolentry}

        \vspace{0.10 cm}
        \begin{onecolentry}
            \begin{highlights}
                \item To Demostrate the real-time stroke assessment algorithm ability on edge device, Currently developing a mobile app using face/hand/box/blocks detection to count number of blocks transferred during Box and Block Test. 
                \item Tools Used: Swift, Vision, CoreML
            \end{highlights}
        \end{onecolentry}


        \vspace{0.2 cm}

        \begin{twocolentry}{
            2023
        }
            \textbf{Custom File System}\end{twocolentry}

        \vspace{0.10 cm}
        \begin{onecolentry}
            \begin{highlights}
                \item Built a UNIX-style file system that supports Amazon S3 services
                \item Tools Used: C++, pthread, exec family
            \end{highlights}
        \end{onecolentry}



    
    \section{Technologies}



        
        \begin{onecolentry}
            \textbf{Languages:} Python, C++, C, Go, Swift, Java, R, HTML, CSS, Javascript, MatLab, Chisel, Lisp, Prolog, godot %, Objective-C, C\#, SQL, JavaScript
        \end{onecolentry}

        \vspace{0.2 cm}

        \begin{onecolentry}
            \textbf{Technologies:} Mediapipe, OpenCV, Keras, PyTorch, TensorFlow Ultralytics, Numpy, ROS, Linux, LiDAR, Cameras % .NET, Microsoft SQL Server, XCode, Interface Builder
        \end{onecolentry}

    \section{Honors and Awards}
        \begin{onecolentry}
            \textbf{Dean's List: } Fall 2021, Winter 2022, Spring 2022, Spring 2024
        \end{onecolentry}


    

\end{document}